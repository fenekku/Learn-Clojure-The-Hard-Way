\chapter{Exercise 5 : Operations on Collections}

Like it is the case for numbers, specific operations can be applied on collections.
Even some collections allow for the same operations to be applied to them! This is why knowing the difference between each type of collection is important especially
in terms of performance.

Type this is in to practice using the most common collection operation:
\begin{code}{ex5.clj}
<< d['code/ex5.clj|pyg|l'] >>
\end{code} 

What you should see:
\begin{code}{Running ex5.clj}
\begin{Verbatim}
$ clojure ex5.clj 
This is the first element 1
This is the rest (2 3)
This is the last element 3
Same thing is possible on a vector: 
 First:  1 rest:  (2 3) Last:  3
Get the nth element (literally):  [Vector]
Another way: [Vector]
Add to the front of a list:  (0 1 2 3)
Dog's name: Fido
# key/value pairs: 2
\end{Verbatim}
\end{code}

Extra credit

\begin{enumerate}
\item{Try different mixes of operations to see which are applicable to which data types}
\item{Look up other operations that can be applied on collections}
\end{enumerate}