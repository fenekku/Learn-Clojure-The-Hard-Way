\chapter{Exercise 4 : Collections}

You have dealt so far with two important data types: numbers and strings. The latter is what programmers call a sequence of characters enclosed in double quotes. You were printing strings out when calling (println "Hello World").

It is now time to look at the four other most important data types in Clojure:
lists, vectors, sets and maps.

Type the following:
\begin{code}{ex4-1.clj}
<< d['code/ex4-1.clj|pyg|l'] >>
\end{code} 

What you should see:

\begin{code}{Running ex4-1.clj}
\begin{Verbatim}
$ clojure ex4-1.clj 
(1 2 3)
(1 2 3)
[1 2 3]
[1 2 3]
[1 2 3]
#{is weird This}
#{is weird This}
#{1 2}
{:key2 value2, :key1 value1}
{:key2 value2, :key1 value1}
\end{Verbatim}
\end{code}

Let's break it down.

{\bf Lists}

A list as its name implies is just a sequence of elements. You can think of lists as being implemented by a linked list under the hood. If you don't know what that means look it up. This is why there are also vectors.

{\bf Vectors}

A vector is also a sequence of elements, but it is akin to an array in C or Ruby. 
Vectors are efficient for random access but not so much for updates.

{\bf Sets}

A set is just like a set in Math: a collection of unique elements. There is no order in a set apriori (but look carefully, can you find the order Clojure uses?) 
Sets are useful for doing set operations and the like. We'll see those soon!

{\bf Maps}

A map is an association of keys with values. They are the equivalent of a dict in Python or a HashMap in Java (humhumhum).




