\chapter{Exercise 0: The Setup}

This exercise has no code.  It is simply the exercise you complete	
to get your computer setup to run Clojure. You should follow these instructions
as exactly as possible.  For example, Mac OSX computers [fill what is the state of Clojure on Mac OSX].
	
	
.. warning::
	
	
    If you do not know how to use PowerShell on Windows or the Terminal on
    OSX or "bash" on Linux then you need to go learn that first.  Zed Shaw has
    a quick crash course at http://cli.learncodethehardway.org/ which is free
    and will teach you the basics of PowerShell and Terminal quickly. Go
    through that then come back here.
	
	
Mac OSX
	
=======
	
	
To complete this exercise, complete the following tasks

(Change based on way it is on Mac OSX - I don't know)

Windows
	
=======

(Change based on way it is on Windows - I don't know)	
	
Linux
	
=====
	
	
Linux is a varied operating system with a bunch of different ways to install software.
	
I'm assuming if you are running Linux then you know how to install packages so here are your instructions:
	
	
. Use your Linux package manager and install the ``gedit`` text editor.
	
. Make sure you can get to ``gedit`` easily by putting it in your window manager's menu.
	
	
   a. Run gedit so we can fix some stupid defaults it has.
   b. Open ``Preferences`` select the ``Editor`` tab.
   c. Change ``Tab width:`` to 4.
   d. Select (make sure a check mark is in) ``Insert spaces instead of tabs``.
   e. Turn on "Automatic indentation" as well.
   f. Open the ``View`` tab turn on "Display line numbers".

	
. Find your "Terminal" program.  It could be called ``Terminal``, ``GNOME Terminal``, ``Konsole``, or ``xterm``.
	
. Put your Terminal in a location you can get to easily as well.
	
. Run your Terminal program.  It won't look like much.
	
. In your Terminal program, run ``clojure``.  You run things in Terminal by just typing their name and hitting RETURN.
	
	
   a. If you run ``clojure`` and it's not there, install it with your package manager. In Ubuntu it looks like this: sudo apt-get install clojure . You will then be prompted to retype the command with the version you want.  *Make sure you install clojure1.3 or above*
	
	
. Hit CTRL-D (\^D) and get out of ``clojure``.
	
. You should be back at a prompt similar to what you had before you typed ``clojure``.  If not find out why.
	
. Learn how to make a directory in the Terminal. Search online for help.
	
. Learn how to change into a directory in the Terminal.  Again search online.
	
. Use your editor to create a file in this directory.  Typically you
	
    will make the file, "Save" or "Save As..", and pick this directory.
	
. Go back to Terminal using just the keyboard to switch windows.  Look it
	
    up if you can't figure it out.
	
. Back in Terminal see if you can list the directory to see your 
	
    newly created file.  Search online for how to list a directory.
	
	
	
Linux: What You Should See
	
--------------------------
	
	
	
.. code-block:: console
	
	
    \$ clojure
	
    Clojure 1.3.0
	user=> 
	
    \$ mkdir mystuff
	
    \$ cd mystuff
	
    \# ... Use gedit here to edit test.txt ...
	
    \$ ls
	
    test.txt
	
    \$ 
	
	
You will probably see a very different prompt, Clojure information, and other stuff but this is the general idea.
	

	
Warnings For Beginners
	
======================
	
	
You are done with this exercise.  This exercise might be hard for you
	
depending on your familiarity with your computer.  If it is difficult,
	
take the time to read and study and get through it, because until you can do
	
these very basic things you will find it difficult to get much programming done.
	
	
If a programmer tells you to use ``vim`` or ``emacs``, tell them, "No."  These
	
editors are for when you are a better programmer.  All you need right now is an
	
editor that lets you put text into a file.  We will use ``gedit``,
	
``TextWrangler``, or ``Notepad++`` (from now on called "the text editor" or "a
	
text editor") because it is simple and the same on all computers.  Professional
	
programmers use these text editors so it's good enough for you starting out.
	
	
A programmer may try to get you to install Python 3 and learn that.  You
	
should tell them, "When all of the python code on your computer is Python 3,
	
then I'll try to learn it."  That should keep them busy for about 10 years.
	
	
A programmer will eventually tell you to use Mac OSX or Linux.  If the programmer
	
likes fonts and typography, they'll tell you to get a Mac OSX computer.  If they
	
like control and have a huge beard, they'll tell you to install Linux.  Again,
	
use whatever computer you have right now that works.  All you need is ``gedit``,
	
a Terminal, and ``python``.
	
	
Finally the purpose of this setup is so you can do three things very reliably
	
while you work on the exercises:
	
	
. *Write* exercises using your text editor, ``gedit`` on Linux, ``TextWrangler`` on OSX, ``Notepad++`` on Windows.
	
. *Run* the exercises you wrote.
	
. *Fix* them when they are broken.
	
. Repeat.
	
	
Anything else will only confuse you, so stick to the plan.