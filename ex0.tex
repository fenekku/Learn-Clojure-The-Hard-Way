\chapter{Exercise 0: The Setup}

This exercise has no code.  It is simply the exercise you complete	
to get your computer setup to run Clojure. You should follow these instructions
as exactly as possible.
	
	
\begin{aside}{Warning}

    If you do not know how to use PowerShell on Windows or the Terminal on
    OSX or "bash" on Linux then you need to go learn that first.  Zed Shaw has
    a quick crash course at http://cli.learncodethehardway.org/ which is free
    and will teach you the basics of PowerShell and Terminal quickly. Go
    through that then come back here.

\end{aside}
	
{\Large Mac OSX}

To complete this exercise, complete the following tasks

(Change based on way it is on Mac OSX - I don't know)

{\Large Windows}

To complete this exercise, complete the following tasks

(Change based on way it is on Windows - I don't know)	
	
{\Large Linux}
	
Linux is a varied operating system with a bunch of different ways to install software.
	
I'm assuming that if you are running Linux then you know how to install packages so here are your instructions:
	
\begin{enumerate}	
\item Use your Linux package manager and install the "gedit" text editor.
\item Make sure you can get to "gedit" easily by putting it in your window manager's menu.
	\begin{enumerate}
	\item Run gedit so we can fix some stupid defaults it has.
   	\item Open "Preferences" select the "Editor" tab.
   	\item Change "Tab width:" to 4.
   	\item Select (make sure a check mark is in) "Insert spaces instead of tabs".
   	\item Turn on "Automatic indentation" as well.
   	\item Open the "View" tab turn on "Display line numbers".
   	\end{enumerate}
	
\item Find your "Terminal" program.  It could be called "Terminal", "GNOME Terminal", "Konsole", or "xterm".
\item Put your Terminal in a location you can get to easily as well.
\item Run your Terminal program.  It won't look like much.
	
\item In your Terminal program, run "clojure".  You run things in Terminal by just typing their name and hitting RETURN.
	\begin{enumerate}
	\item If you run "clojure" and it's not there, install it with your package manager. In Ubuntu it looks like this: sudo apt-get install clojure . You will then be prompted to retype the command with the version you want.  \ident{Make sure you install clojure1.3 or above}
	\end{enumerate}
	
\item Hit CTRL-D (\^D) and get out of "clojure". You should be back at a prompt similar to what you had before you typed "clojure".  If not find out why.
\item Learn how to make a directory in the Terminal. Search online for help.
\item Learn how to change into a directory in the Terminal.  Again search online.
\item Use your editor to create a file in this directory.  Typically you
will make the file, "Save" or "Save As...", and pick this directory.
\item Go back to Terminal using just the keyboard to switch windows.  Look it up if you can't figure it out.
\item Back in Terminal see if you can list the directory to see your newly created file.  Search online for how to list a directory.
\end{enumerate}	
	
{\large Linux: What You Should See}
	
\begin{code}{Terminal output}
\begin{Verbatim}
$ clojure
Clojure 1.3.0
user=> 
# EXIT with ^D
$ mkdir mystuff
$ cd mystuff
# ... Use gedit here to edit test.txt ...
$ ls
  test.txt
\end{Verbatim}
\end{code}

	
You will probably see a very different prompt, Clojure information, and other stuff but this is the general idea.
	

{\Large Warnings For Beginners}
	
You are done with this exercise.  This exercise might be hard for you
depending on your familiarity with your computer.  If it is difficult,
take the time to read and study and get through it, because until you can do
these very basic things you will find it difficult to get much programming done.
	
If a programmer tells you to use "vim" or "emacs", tell them, "No."  These
editors are for when you are a better programmer.  All you need right now is an
editor that lets you put text into a file.  We will use "gedit",
"TextWrangler", or "Notepad++" (from now on called "the text editor" or "a
text editor") because it is simple and the same on all computers.  Professional
programmers use these text editors so it's good enough for you starting out.
	
A programmer will eventually tell you to use Mac OSX or Linux.  If the programmer
likes fonts and typography, they'll tell you to get a Mac OSX computer.  If they
like control and have a huge beard, they'll tell you to install Linux.  Again,
use whatever computer you have right now that works.  All you need is "gedit",
a Terminal, and "clojure".
	
Finally the purpose of this setup is so you can do three things very reliably
while you work on the exercises:
	
\begin{enumerate}
\item \ident{Write} exercises using your text editor, "gedit" on Linux, "TextWrangler" on Mac OSX, "Notepad++" on Windows. Don't copy-paste.
\item \ident{Run} the exercises you wrote.
\item \ident{Fix} them when they are broken.
\item Repeat.
\end{enumerate}
	
Anything else will only confuse you, so stick to the plan.